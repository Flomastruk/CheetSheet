\documentclass[12pt]{article}


\usepackage{amssymb,amsmath,amsthm}

\usepackage{wrapfig,tikz, tikz-cd}
\usetikzlibrary{arrows, arrows, calc, decorations.markings, automata,calc}

\newcommand{\dd}{\partial}
\newcommand{\argmax}{\operatorname{argmax}}
\newcommand{\cut}{\operatorname{cut}}
\newcommand{\Span}{\operatorname{Span}}
\newcommand{\RR}{\mathbb{R}}
\newcommand{\KK}{\mathbb{K}}
\newcommand{\rr}{\text{r}}
\newcommand{\al}{\alpha}
\newcommand{\be}{\beta}
\newcommand{\eps}{\varepsilon}
\newcommand{\ph}{\varphi}
\newcommand{\ZZ}{\mathbb{Z}}
\newcommand{\BB}{\mathfrak{B}}
\newcommand{\MM}{\mathfrak{M}}
\newcommand{\kk}{\textbf{k}}

\usepackage{hyperref}
\hypersetup{
    colorlinks=true,
    linkcolor=blue,
    filecolor=magenta,
    urlcolor=cyan,
}

\usepackage{tcolorbox}

\definecolor{tangoBlack1}{RGB}{0,0,0}
\definecolor{tangoGrey1}{RGB}{210,210,210}

\makeatletter
\newtcbox{\keywordmin}{
  on line,
  fontupper=\sffamily,
  boxrule=1pt,
  arc=1pt,
  coltext=tangoBlack1,
  colback=tangoGrey1,
  colframe=tangoGrey1,
  boxsep=0pt,
  left=2pt,right=2pt,top=2pt,bottom=2pt
}
\makeatother

\begin{document}





\title{Git cheat sheet}
\author{Efim Abrikosov}
\maketitle


\section{Main commands}
\begin{itemize}
  \item \keywordmin{git help \emph{functionname}} --- display help information about Git
  \item \keywordmin{git init \emph{[-{}-template=templatedirectory]}} --- create an empty Git repository or reinitialize an existing one. Files and directories in the template directory whose name do not start with a dot will be copied to the directory after it is created.
  \item \keywordmin{git clone  \emph{repository [directory]}} --- clone a repository into a new directory. Optionally supply the name of a new directory to clone into.
  \item \keywordmin{git add \emph{[filename]}} --- add file contents to the index
  \item \keywordmin{git commit  \emph{[-a][-m text]}} --- record changes to the repository
  \item \keywordmin{git diff  \emph{[cached]}}
  \item \keywordmin{git status }
\end{itemize}




\subsection{Base workflow cases}


\section{Setting up a Google Cloud project}
\begin{itemize}
  \item Log in your google account
  \item Browse to cloud.google.com
  \item Click on ``Go to Console''
  \item Go to Navigation menu (three horizontal lines in the top left corner)
  \item Select ``Compute Engine''
  \item Click ``Create''
  \item Select ``Allow full access to all Cloud APIs''
  \item Now click ``Create''
% SSH into
  \item Click on ``SSH'' field in the VM list to open the console
  \item To update the system configuration type ``sudo apt-get update''
  \item To install Git type in ``sudo apt-get -y -qq install git''
% Creating a Storage Bucket
  \item Go to Navigation menu
  \item Select ``Storage''
  \item Click ``Create bucket''
  \item Select appropriate settings
  \item Now click ``Create''
% Storing VM files to the Bucket
  \item In console type gsutil cp [filename] gs:://[bucketname]/[pathname]
  \item To publish cloud storage files to the web run gsutil acl ch -u AllUsers:R gs://[bucketname]/[pathname]
% Launching Cloud Datalab
  \item To launch Cloud Datalab, open a Cloud Shell in the Platform page (the icon is in the top right corner)
  \item In Cloud Shell type ``gcloud compute zones list''
  \item In Cloud Shell type ``datalab create mydatalabvm --zone [zonename]''
  \item Creating Datalab VM may take several minutes
  \item Click on ``Web Preview'' button in the top of the Cloud Shell and change port to 8081
% Invoking BigQuery
  \item Go to Navigation menu
  \item Select ``BigQuery''
  \item In More Options click ``Query settings''
  \item Under Additional Settings ensure that Legacy is not enabled
  \item In the query textbox type necessary SQL commands to extract data from big datasets
% accessing data from DataLab
  \item Create a notebook in Datalab
  \item Define a valid query string in the notebook
  \item Use the following logic:
  \begin{enumerate}
    \item import google.datalab.bigquery as bq
    \item df = bq.Query(query).execute().result().to\_dataframe()
    \item df.head()
  \end{enumerate}
% invoking Google pre-trained Machine Learning API
\item Launch Cloud Datalab
\item To download git repository contents use the logic
\begin{enumerate}
  \item \%bash
  \item git clone [repositoryaddress] m -rf [pathname]
\end{enumerate}
\item Select APIs\&Services from Cloud Platform Navigation Menu
\item Click ``Library'' and search for required API (e.g. Cloud Vision API, Translate API, Speech API, or Natural Language API)
\item Click ``Enable'' if necessary
\item In APIs\&Services click ``Credentials'' and create ``API key'' if necessary. This key will be used in Datalab code to invoke various APIs
%Invoking ML APIs from Datalab
\item In Datalab use APIKEY generated in credentials as ``developerKey'' parameter in Datalab code
\item In Datalab, run
``!pip install --upgrade google-api-python-client''
\begin{itemize}
  \item Translate API
  \begin{enumerate}
    \item from googleapiclient.discovery import build
    \item service = build('translate', 'v2', developerKey=APIKEY)
  \end{enumerate}
  \item Vision API
  \begin{enumerate}
    \item import base64
    \item vservice = build('vision', 'v1', developerKey=APIKEY)
  \end{enumerate}
  \item Natural Language API
  \begin{enumerate}
    \item lservice = build('language', 'v1beta1', developerKey=APIKEY)
  \end{enumerate}
  \item Speech API
  \begin{enumerate}
    \item build('speech', 'v1beta1', developerKey=APIKEY)
  \end{enumerate}
\end{itemize}
% using BigQuery from Datalab

  \item Launch Cloud Datalab
  \item Use the following logic
  \begin{itemize}
    \item google.datalab.bigquery as bq
    \item qry = \textsc{\char13}\textsc{\char13}\textsc{\char13} SELECT * FROM ...\textsc{\char13}\textsc{\char13}\textsc{\char13}
    \item bq.Query(qry).execute().result().to\_dataframe()
  \end{itemize}

\end{itemize}


\section{Useful links}

\begin{itemize}
\item\href{https://github.github.com/training-kit/downloads/github-git-cheat-sheet.pdf}{Git cheet sheet}
\end{itemize}

\end{document}
