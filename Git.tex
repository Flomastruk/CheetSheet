\documentclass[12pt]{article}


\usepackage{amssymb,amsmath,amsthm}

\usepackage{wrapfig,tikz, tikz-cd}
\usetikzlibrary{arrows, arrows, calc, decorations.markings, automata,calc}

\newcommand{\dd}{\partial}
\newcommand{\argmax}{\operatorname{argmax}}
\newcommand{\cut}{\operatorname{cut}}
\newcommand{\Span}{\operatorname{Span}}
\newcommand{\RR}{\mathbb{R}}
\newcommand{\KK}{\mathbb{K}}
\newcommand{\rr}{\text{r}}
\newcommand{\al}{\alpha}
\newcommand{\be}{\beta}
\newcommand{\eps}{\varepsilon}
\newcommand{\ph}{\varphi}
\newcommand{\ZZ}{\mathbb{Z}}
\newcommand{\BB}{\mathfrak{B}}
\newcommand{\MM}{\mathfrak{M}}
\newcommand{\kk}{\textbf{k}}
\usepackage{tcolorbox}

\definecolor{tangoBlack1}{RGB}{0,0,0}
\definecolor{tangoGrey1}{RGB}{210,210,210}

\makeatletter
\newtcbox{\keywordmin}{
  on line,
  fontupper=\sffamily,
  boxrule=1pt,
  arc=1pt,
  coltext=tangoBlack1,
  colback=tangoGrey1,
  colframe=tangoGrey1,
  boxsep=0pt,
  left=2pt,right=2pt,top=2pt,bottom=2pt
}
\makeatother

\begin{document}





\title{Git cheet sheet}
\author{Efim Abrikosov}
\maketitle


\section{Main commands}
\begin{itemize}
  \item \keywordmin{git help \emph{functionname}} --- display help information about Git
  \item \keywordmin{git init \emph{[-{}-template=templatedirectory]}} --- create an empty Git repository or reinitialize an existing one. Files and directories in the template directory whose name do not start with a dot will be copied to the directory after it is created.
  \item \keywordmin{git clone  \emph{repository [directory]}} --- clone a repository into a new directory. Optionally supply the name of a new directory to clone into.
  \item \keywordmin{git add \emph{[filename]}} --- add file contents to the index
  \item \keywordmin{git commit  \emph{[-a][-m text]}} --- record changes to the repository
  \item \keywordmin{git diff  \emph{[cached]}}
  \item \keywordmin{git status }
\end{itemize}




\section{Base workflow cases}


\section{Useful links}


\end{document}
