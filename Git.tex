\documentclass[12pt]{article}

\usepackage{amssymb,amsmath,amsthm}
\usepackage{listings}

\usepackage{wrapfig,tikz,tikz-cd}
\usetikzlibrary{arrows, arrows, calc, decorations.markings, automata,calc}

\newcommand{\dd}{\partial}
\newcommand{\argmax}{\operatorname{argmax}}
\newcommand{\cut}{\operatorname{cut}}
\newcommand{\Span}{\operatorname{Span}}
\newcommand{\RR}{\mathbb{R}}
\newcommand{\KK}{\mathbb{K}}
\newcommand{\rr}{\text{r}}
\newcommand{\al}{\alpha}
\newcommand{\be}{\beta}
\newcommand{\eps}{\varepsilon}
\newcommand{\ph}{\varphi}
\newcommand{\ZZ}{\mathbb{Z}}
\newcommand{\BB}{\mathfrak{B}}
\newcommand{\MM}{\mathfrak{M}}
\newcommand{\kk}{\textbf{k}}

\usepackage{hyperref}
\hypersetup{
    colorlinks=true,
    linkcolor=blue,
    filecolor=magenta,
    urlcolor=cyan,
}

\usepackage{tcolorbox}

\definecolor{tangoBlack1}{RGB}{0,0,0}
\definecolor{tangoGrey1}{RGB}{210,210,210}
\definecolor{dkgreen}{rgb}{0,0.6,0}
\definecolor{gray}{rgb}{0.5,0.5,0.5}
\definecolor{mauve}{rgb}{0.58,0,0.82}
\definecolor{deepblue}{rgb}{0,0,0.5}
\definecolor{deepred}{rgb}{0.6,0,0}
\definecolor{deepgreen}{rgb}{0,0.5,0}


%https://tex.stackexchange.com/questions/83882/how-to-highlight-python-syntax-in-latex-listings-lstinputlistings-command
% Python style for highlighting
\DeclareFixedFont{\ttb}{T1}{txtt}{bx}{n}{12} % for bold
\DeclareFixedFont{\ttm}{T1}{txtt}{m}{n}{12}  % for normal

%bash
\newcommand\bashstyle{\lstset{
language=bash,
basicstyle=\small\ttfamily,
commentstyle=\color{red},
keywordstyle=\color{deepblue},
showstringspaces=false
}}

% Bash environment
\lstnewenvironment{bash}[1][]
{
\bashstyle
\lstset{#1}
}
{}

% bash for inline
\newcommand\bashinline[1]{{\bashstyle\lstinline!#1!}}


\newcommand\pythonstyle{\lstset{
language=Python,
basicstyle=\ttm,
otherkeywords={self},             % Add keywords here
keywordstyle=\ttb\color{deepblue},
emph={MyClass,__init__},          % Custom highlighting
emphstyle=\ttb\color{deepred},    % Custom highlighting style
stringstyle=\color{deepgreen},
frame=tb,                         % Any extra options here
showstringspaces=false            %
}}

% Python environment
\lstnewenvironment{python}[1][]
{
\pythonstyle
\lstset{#1}
}
{}

% Python for inline
\newcommand\pythoninline[1]{{\pythonstyle\lstinline!#1!}}


\makeatletter
\newtcbox{\keywordmin}{
  on line,
  fontupper=\sffamily,
  boxrule=1pt,
  arc=1pt,
  coltext=tangoBlack1,
  colback=tangoGrey1,
  colframe=tangoGrey1,
  boxsep=0pt,
  left=2pt,right=2pt,top=2pt,bottom=2pt
}
\makeatother


\begin{document}

\title{Git cheat sheet}
\author{Efim Abrikosov}
\maketitle


\section{Main commands}
\subsection{Git}
\begin{itemize}
  \item \keywordmin{git help \emph{commandname}} --- display help information about Git command
  \item \keywordmin{git init \emph{[-{}-template=templatedirectory]}} --- create an empty Git repository or reinitialize an existing one. Files and directories in the template directory whose name do not start with a dot will be copied to the directory after it is created.
  \item \keywordmin{git clone \emph{repository [directory]}} --- clone a repository into a new directory. Optionally supply the name of a new directory to clone into.
  \item \keywordmin{git remote add \emph{name} \emph{url}} --- add a remote (e.g. named ``origin'') for the repository at a specified web address
  \item \keywordmin{git add \emph{[filename]}} --- updates the index using the current content found in the working tree, to prepare the content staged for the next commit. If filenames are specified, then only given files are staged
  \item \keywordmin{git commit  \emph{[-a][-m text]}} --- record changes to the repository
  \item \keywordmin{git diff  \emph{[cached]}}
  \item \keywordmin{git status}
\end{itemize}


\subsection{Anaconda environment management}
\begin{itemize}
  \item \keywordmin{conda \emph{commandname} -h} --- display help information about command
  \item \keywordmin{conda env list} --- list all anaconda environments
  \item \keywordmin{conda list -n \emph{myenv}} --- list packages installed in the environment
  \item \keywordmin{conda activate \emph{[envname]}} --- activate anaconda environment
  \item \keywordmin{conda deactivate} --- deactivate current environment
  \item \keywordmin{conda create \emph{[-{}-clone envname]} \emph{-n newenvname}} --- create a new environment
  \item \keywordmin{conda remove -{}-name \emph{myenv} -{}-all} --- remove existing environment
  \item \keywordmin{conda install \emph{[-{}-name myenv]} \emph{packagename [= version]} } --- install package, if no environment is specified, the package is installed in the current environment
  \item \keywordmin{conda search \emph{packagename}} --- search for packages with a given name
  \item \keywordmin{install -{}-revision \emph{revnum}} --- revert to a given revision
  \item \keywordmin{conda remove \emph{[-n myenv]} \emph{packagename}} --- remove package, if no environment is specified, the package is removed from the current environment
\end{itemize}

\subsection{Base workflow cases}


\section{Setting up a Google Cloud project}
\begin{itemize}
  \item Log in your google account
  \item Browse to cloud.google.com
  \item Click on ``Go to Console''
  \item Go to Navigation menu (three horizontal lines in the top left corner)
  \item Select ``Compute Engine''
  \item Click ``Create''
  \item Select ``Allow full access to all Cloud APIs''
  \item Now click ``Create''
% SSH into
  \item Click on ``SSH'' field in the VM list to open the console
  \item To update the system configuration type ``sudo apt-get update''
  \item To install Git type in ``sudo apt-get -y -qq install git''
% Creating a Storage Bucket
  \item Go to Navigation menu
  \item Select ``Storage''
  \item Click ``Create bucket''
  \item Select appropriate settings
  \item Now click ``Create''
% Storing VM files to the Bucket
  \item In console type gsutil cp [filename] gs:://[bucketname]/[pathname]
  \item To publish cloud storage files to the web run gsutil acl ch -u AllUsers:R gs://[bucketname]/[pathname]
% Launching Cloud Datalab
  \item To launch Cloud Datalab, open a Cloud Shell in the Platform page (the icon is in the top right corner)
  \item In Cloud Shell type ``gcloud compute zones list''
  \item In Cloud Shell type ``datalab create mydatalabvm --zone [zonename]''
  \item Creating Datalab VM may take several minutes
  \item Click on ``Web Preview'' button in the top of the Cloud Shell and change port to 8081
% Invoking BigQuery
  \item Go to Navigation menu
  \item Select ``BigQuery''
  \item In More Options click ``Query settings''
  \item Under Additional Settings ensure that Legacy is not enabled
  \item In the query textbox type necessary SQL commands to extract data from big datasets
% accessing data from DataLab
  \item Create a notebook in Datalab
  \item Define a valid query string in the notebook
  \item Use the following logic:
  \begin{enumerate}
    \item import google.datalab.bigquery as bq
    \item df = bq.Query(query).execute().result().to\_dataframe()
    \item df.head()
  \end{enumerate}
% invoking Google pre-trained Machine Learning API
\item Launch Cloud Datalab
\item To download git repository contents use the logic
\begin{enumerate}
  \item \%bash
  \item git clone [repositoryaddress] m -rf [pathname]
\end{enumerate}
\item Select APIs\&Services from Cloud Platform Navigation Menu
\item Click ``Library'' and search for required API (e.g. Cloud Vision API, Translate API, Speech API, or Natural Language API)
\item Click ``Enable'' if necessary
\item In APIs\&Services click ``Credentials'' and create ``API key'' if necessary. This key will be used in Datalab code to invoke various APIs
% Invoking ML APIs from Datalab
\item In Datalab use APIKEY generated in credentials as ``developerKey'' parameter in Datalab code
\item In Datalab, run
``!pip install --upgrade google-api-python-client''
\begin{itemize}
  \item Translate API
  \begin{enumerate}
    \item from googleapiclient.discovery import build
    \item service = build('translate', 'v2', developerKey=APIKEY)
  \end{enumerate}
  \item Vision API
  \begin{enumerate}
    \item import base64
    \item vservice = build('vision', 'v1', developerKey=APIKEY)
  \end{enumerate}
  \item Natural Language API
  \begin{enumerate}
    \item lservice = build('language', 'v1beta1', developerKey=APIKEY)
  \end{enumerate}
  \item Speech API
  \begin{enumerate}
    \item build('speech', 'v1beta1', developerKey=APIKEY)
  \end{enumerate}
\end{itemize}
% using BigQuery from Datalab

  \item Launch Cloud Datalab
  \item Use the following logic
  \begin{itemize}
    \item google.datalab.bigquery as bq
    \item qry = \textsc{\char13}\textsc{\char13}\textsc{\char13} SELECT * FROM ...\textsc{\char13}\textsc{\char13}\textsc{\char13}
    \item bq.Query(qry).execute().result().to\_dataframe()
  \end{itemize}

\end{itemize}

\subsection{Setting up a new Google Compute Engine VM}

Better, follow instructions \href{https://cloud.google.com/python/setup}{here}.

Make sure that suitable python version is installed:
\keywordmin{python -{}-version}
\keywordmin{python3 -{}-version}

If python or python3 is not installed on the machine, follow instructions \href{https://www.vultr.com/docs/upgrade-python-on-debian}{here}. Note that python version must be compatible with your distribution system (Ubuntu/Debian/etc...)

Install `pip' package management system with distribution:
\keywordmin{sudo apt install python3-pip}

Check that pip is installed
\keywordmin{python3 -m pip -{}-version}

Install `virtualenv' package. WARNING: for now do not simply use \keywordmin{pip3 install [package name]}, rather rely on \keywordmin{python -m pip install [package name] --user}. This ensures that package management is controlled by pip and not by OS distribution system (\href{https://github.com/pypa/pip/issues/5599}{some details}).
\keywordmin{wh}

\subsection{Using Google Cloud AI Platform to train models}
Use your local environment to develop a model. Organize it in the form of a python package with \emph{task\_script.py} script that can be launched locally using the following minimal example

\begin{bash}
python -m task_package.task_script /
  --job-dir <path to job dir> [optional arguments]
\end{bash}

Typically the task script will contain instructions to parse command line arguments, define a model and data pipeline, train the model record checkpoints and finally store model weights,

When the package is ready it can be seamlessly transferred to run on Google Cloud AI Platform with the following minimal example:

\begin{bash}
gcloud ai-platform jobs submit training $JOB_NAME /
  --package-path $PATH_TO_PACKAGE /
  --module-name task_script /
  --region $REGION /
  --job-dir $JOB_DIR /
  --stream-logs /
  -- [optional arguments accepted by python task script]
\end{bash}
Supplied arguments are as follows:
\begin{itemize}\setlength\itemsep{-5pt}
  \item[\tiny$\bullet$] \bashinline{\$JOB_NAME} unique name assigned to a job. List of jobs can be viewd with \bashinline{gcloud ai-platform jobs list}
  \item[\tiny$\bullet$] \bashinline{--package-path} local path to the package (e.g. \bashinline{task\_package/})
  \item[\tiny$\bullet$] \bashinline{--module-name} name of task script in the package (e.g. \bashinline{task\_script})
  \item[\tiny$\bullet$] \bashinline{--region} name of gcloud compute region (e.g. \bashinline{us-east1}). List of regions can be viewd with \bashinline{gcloud compute regions list}
  \item[\tiny$\bullet$] \bashinline{\$JOB_DIR} valid path accessible by AI Platform. For instance, can use a directory in a Google Cloud Storage Bucket (list project buckets with \bashinline{gsutil ls})
  \item[\tiny$\bullet$] \bashinline{--stream-logs} view script  stdout in command line during the execution
\end{itemize}
Further control over execution can be fine-tuned with the following useful AI Platform options:
\begin{itemize}\setlength\itemsep{-5pt}
  \item[\tiny$\bullet$] \bashinline{python-version 3.7}
  \item[\tiny$\bullet$] \bashinline{--runtime-version 2.1} use TensorFlow version 2.1 for execution (\href{https://cloud.google.com/ai-platform/training/docs/runtime-version-list}{list} of available runtime versions and packages within)
\end{itemize}


\section{Useful links}

\begin{itemize}
\item\href{https://github.github.com/training-kit/downloads/github-git-cheat-sheet.pdf}{Git cheet sheet}
\item\href{https://docs.conda.io/projects/conda/en/4.6.0/commands.html}{Anaconda command list}
\end{itemize}

\end{document}
